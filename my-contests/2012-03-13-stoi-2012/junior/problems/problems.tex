\documentclass[a4paper]{article}

\usepackage{amsmath}
\usepackage{amssymb}
\usepackage{cmap}
\usepackage{geometry}
\usepackage{hyperref}
\usepackage{indentfirst}
\usepackage{xeCJK}

\geometry{margin=1in}

\setCJKmainfont[BoldFont={SimHei}]
{SimSun}
\setCJKmonofont{FangSong}

\title{2012年汕头市青少年信息学奥林匹克决赛\\
普及组}
%\author{ftiasch}
\date{2012年4月15日}

\renewcommand{\thesection}{第\arabic{section}题}
\renewcommand{\thesubsection}{}

\newcommand{\problem}{\section}
\newcommand{\inputformat}{\subsection{输入格式}}
\newcommand{\outputformat}{\subsection{输出格式}}
\newcommand{\sample}[2]{
\subsection{输入输出样例}
\begin{tabular}{|l|l|}
    \hline
    输入 & 输出 \\
    \hline
    \begin{minipage}[t]{200pt}
        \begin{ttfamily} #1 \end{ttfamily}
    \end{minipage} &
    \begin{minipage}[t]{200pt}
        \begin{ttfamily} #2 \end{ttfamily}
    \end{minipage} \\
    \hline
\end{tabular}
\vspace{1ex}
\par
}
\newcommand{\dataset}{\subsection{数据范围}}

\begin{document}

\maketitle

\begin{enumerate}
\item 参赛选手提交\textbf{源文件},文件名为对应题目英文加上语言后缀。例如,test一题的提交应该名为test.pas, test.c或者test.cpp。\textbf{不要}在提交目录下建立子目录。

\item 程序需要使用\textbf{文件}输入输出,输入输出文件名为对应题目英文加上文件类型。例如,test一题的输入输出文件分别为test.in和test.out。

\item 时间限制原则上至少是标准程序运行时间的$2$倍,空间限制为$256$M。

\item 最终解释权归出题人所有。
\end{enumerate}

\problem{排序(sort)}

ftiasch和nm是好朋友。nm的成绩很差,以至于GPA(平均绩点)在系内倒数。系内一共有$N$位同学,每位同学有自己的GPA,以及已修学分数,定义$\textrm{GPT} = \textrm{GPA} \times \textrm{已修学分数}$。ftiasch为了帮助nm提高成绩,给nm提了一个要求:新学期的GPA要超过系内排名第$K$位的同学。

为了帮助理解,给出一个例子:

\begin{center}
\begin{tabular}{|l|l|l|l|}
\hline
排名 & GPA & 已修学分数 & GPT \\
\hline
1 & 3.8 & 21 & 79.8 \\
\hline
2 & 3.7 & 23 & 85.1 \\
\hline
3 & 3.65 & 20 & 73 \\
\hline
4(跟3一样) & 3.65 & 18 & 65.7 \\
\hline
5 & 3.3 & 22 & 72.6 \\
\hline
\end{tabular}
\end{center}

现在给出系里面每位同学的GPT(只有一位小数),以及他们的已修学分。你需要帮助nm把排名第$K$位的同学的GPA求出来。

\inputformat{}

第$1$行,$2$个整数$N$, $K$。第$2 \sim (N + 1)$行,每行$1$个非负实数和$1$个整数,分别表示GPT和已修学分数,注意所有同学的学分都在$[1, 250]$的范围。

\outputformat{}

第$1$行,$1$个实数,表示排名第$K$同学的GPA,保留$2$位小数输出。

\sample{
5 3 \\
73 20 \\
79.8 21 \\
72.6 22 \\
85.1 23 \\
65.7 18 \\
}{
3.65 \\
}

\dataset{}

\begin{itemize}
    \item 对于$50\%$的数据,$1 \leq N \leq 100$。

    \item 对于$100\%$的数据,$1 \leq K \leq N \leq 100,000$,GPT小数点后至多$1$位,GPA至多$4.0$。
\end{itemize}

\problem{求和(sum)}

ftiasch有很多糖果,分成了$N$堆,排成一列。ftiasch说,如果nm能迅速求出第$L$堆到第$R$堆一共有多少糖果,就把这些糖果都给他。

现在给出每堆糖果的数量,以及每次询问的$L$和$R$,你需要帮助nm,把每次询问的结果求出来。注意,你不需要考虑糖果被nm取走的情况。

\inputformat{}

第$1$行,$2$个整数$N$, $M$, 分别表示堆数和询问数量。第$2$行,$N$个整数$A_i$,表示第$i$堆糖果的数量。第$3 \sim (M + 2)$行,每行$2$个整数$L_i$, $R_i$,表示第$i$个询问是$[L_i, R_i]$。

\outputformat{}

$M$行,对于每个询问,输出对应的和。

\sample{
5 5 \\
1 2 3 4 5 \\
1 5 \\
2 4 \\
3 3 \\
1 3 \\
3 5 \\
}{
15 \\
9 \\
3 \\
6 \\
12 \\
}

\dataset{}

\begin{itemize}
    \item 对于$50\%$的数据,$1 \leq N, M \leq 100$。

    \item 对于$100\%$的数据,$1 \leq N, M \leq 100,000$,$0 \leq A_i \leq 10,000$,$1 \leq L_i \leq R_i \leq N$。
\end{itemize}

\problem{数数(count)}

ftiasch开发了一个奇怪的游戏,这个游戏的是这样的:一个长方形,被分成$N$行$M$列的格子,第$i$行第$j$列的格子记为$(i, j)$,就是说,左上角的格子是$(1, 1)$,右下角的格子是$(N, M)$。开始的时候,nm在$(1, 1)$,他需要走到$(N, M)$。每一步,nm可以走到正右方或者正下方的一个格子。具体地说,如果nm现在在$(x, y)$,那么他可以走到$(x, y + 1)$或$(x + 1, y)$。当然,nm不能走出离开这个长方形。

每个格子有积分,用一个$1 \sim 10$的整数表示。经过这个格子,就会获取这个格子的积分(起点和终点的积分也计算)。通过的方法是:到达$(N, M)$的时候,积分恰好为$P$。

现在给出这个长方形每个格子的积分,你需要帮助nm,求出从起点走到终点,积分为$P$的线路有多少条。

\inputformat{}

第$1$行,$3$个整数$N$, $M$, $P$。接下来$N$行,每行$M$个整数$A_{ij}$,表示格子$(i, j)$的积分。

\outputformat{}

第$1$行,$1$个整数,表示积分为$P$线路的数量。因为数值太大,你只需要输出结果除以$(10^9 + 7)$的余数。

\sample{
3 3 9 \\
2 2 1 \\
2 2 2 \\
1 2 2 \\
}{
2 \\
}

\dataset{}

\begin{itemize}
    \item 对于$50\%$的数据,$1 \leq N, M \leq 10$。

    \item 对于$100\%$的数据,$1 \leq N, M \leq 100$,$0 \leq A_{ij} \leq 10$。
\end{itemize}

\problem{步行(walk)}

ftiasch又开发了一个奇怪的游戏,这个游戏是这样的:有$N$个格子排成一列,每个格子上有一个数字,第$i$个格子的数字记为$A_i$。这个游戏有2种操作:

\begin{enumerate}
    \item 如果现在在第$i$个格子,则可以跳到第$A_i$个格子。

    \item 把某个$A_i$增加或减少$1$。
\end{enumerate}

nm开始在第$1$个格子,他需要走到第$N$个格子才能通关。现在他已经头昏脑涨啦,需要你帮助他求出,从起点到终点最少需要多少次操作。

\inputformat{}

第$1$行,$1$个整数$N$。第$2$行,$N$个整数$A_i$。

\outputformat{}

第$1$行,$1$个整数,表示最少的操作次数。

\sample{
5 \\
3 4 2 5 3 \\
}{
3 \\
}

\dataset{}

\begin{itemize}
    \item 对于$30\%$的数据,$1 \leq N \leq 10$。

    \item 对于$60\%$的数据,$1 \leq N \leq 1,000$。

    \item 对于$100\%$的数据,$1 \leq N \leq 100,000$,$1 \leq A_i \leq N$。
\end{itemize}
\end{document}
