\documentclass[a4paper]{article}

\usepackage{amsmath}
\usepackage{amssymb}
\usepackage{cmap}
\usepackage{geometry}
\usepackage{hyperref}
\usepackage{indentfirst}
\usepackage{xeCJK}

\geometry{margin=1in}

\setCJKmainfont[BoldFont={SimHei}]
{SimSun}
\setCJKmonofont{FangSong}

\title{NOIp'2012 Training Contest \#2}
\author{ftiasch}
\date{\today}

\renewcommand{\thesection}{第\arabic{section}题}
\renewcommand{\thesubsection}{}

\newcommand{\problem}{\section}
\newcommand{\inputformat}{\subsection{输入格式}}
\newcommand{\outputformat}{\subsection{输出格式}}
\newcommand{\sample}[2]{
\subsection{输入输出样例}
\begin{tabular}{|l|l|}
    \hline
    输入 & 输出 \\
    \hline
    \begin{minipage}[t]{200pt}
        \begin{ttfamily} #1 \end{ttfamily}
    \end{minipage} &
    \begin{minipage}[t]{200pt}
        \begin{ttfamily} #2 \end{ttfamily}
    \end{minipage} \\
    \hline
\end{tabular}
\vspace{1ex}
\par
}
\newcommand{\dataset}{\subsection{数据范围}}

\begin{document}

\maketitle

\problem{数列(sequence)}

设递推数列$\{A_n\}$满足:
\[\left\{\begin{array}{c}
A_0 = a \\
A_{n + 1} = (A_n)^2 + A_n + 1 \\
\end{array}\right.\]

给出$a$, $N$,求$A_N \bmod 10003$的值。

\inputformat

$2$个整数$a$和$N$。

\outputformat

$1$个整数,表示$A_N \bmod 10003$的值。

\sample{
0 2 \\
}{
3 \\
}

\dataset
\begin{itemize}
    \item 对于$50\%$的数据,$0 \leq N \leq 10^3$。

    \item 对于$100\%$的数据,$0 \leq N \leq 10^9, 0 \leq a < 10003$。
\end{itemize}

\problem{乘积(product)}

现有数列$A_1, A_2, \ldots, A_N$,其中$A_1$是偶数,$A_2, \ldots, A_N$是奇数,且$A_2 \leq A_3 \leq \cdots \leq A_N$。定义$B = \{A_i \times A_j | 1 \leq i < j \leq N\}$,显然$B$中有$\frac{N(N - 1)}{2}$个元素。

给出$B$,求$A$。

\inputformat

第$1$行,$1$个整数$N$。

第$2$行,$\frac{N(N - 1)}{2}$个整数$B_1, B_2, \ldots, B_{\frac{N(N - 1)}{2}}$。

\outputformat

$N$个整数,表示$A_1, A_2, \ldots, A_N$。

\sample{
3 \\
6 9 6 \\
}{
2 3 3 \\
}

\dataset
\begin{itemize}
    \item 对于$30\%$的数据,$3 \leq N \leq 5$。
    \item 对于$100\%$的数据,$3 \leq N \leq 100$, $1 \leq B_i \leq 10^9$。输入数据保证有解。
\end{itemize}

\problem{队列(queue)}

维护双端队列$Q$,支持$4$种操作:
\begin{itemize}
    \item \texttt{pushBack(e)}: 在队尾插入元素$e$
    \item \texttt{pushFront(e)}: 在队首插入元素$e$
    \item \texttt{popBack()}: 删除队尾元素
    \item \texttt{query(k)}: 设当前队列为$Q_1, Q_2, \ldots, Q_N$,再设$P_i = \max\{Q_1, Q_2, \ldots, Q_i\}$,询问$\{P_1, P_2, \ldots, P_N\}$中第$k$大元素的值。
\end{itemize}

\inputformat

第$1$行,$1$个整数$M$,表示操作的数量。

第$2$到$M + 1$行,每行$1$个字符串,表示每个操作,格式都是\texttt{[操作类型] [参数]},类型$1, 2, 3, 4$分别对应\texttt{pushBack, pushFront, popBack, query}。

\outputformat

对于每个询问操作,输出对应的结果。若第$K$大元素不存在,则输出``-1"。

\sample{
7 \\
1 1 \\
1 2 \\
4 1 \\
4 2 \\
3 \\
2 3 \\
4 2 \\
}{
2 \\
1 \\
-1 \\
}

\dataset
\begin{itemize}
    \item 对于$30\%$的数据,$1 \leq M \leq 10^3$。

    \item 对于$100\%$的数据,$1 \leq M \leq 2 \times 10^5, 0 \leq e \leq 10^9$。任何时刻队列中不会有相同的元素。

\end{itemize}

\end{document}
